\documentclass[titlepage,11pt,a4paper]{article}
\usepackage[utf8]{inputenc}
\usepackage[english]{babel}
\usepackage{graphicx}
\usepackage{float}
\usepackage[hidelinks]{hyperref}

\hypersetup{
  pdfinfo={
    Author={SML},
    Title={Developer's Guide},
    Creator={LaTeX}
  }
}

\title{Developer's Guide}
\author{SML}


\begin{document}
\maketitle
\tableofcontents
\newpage

\section{Introduction}
% Short about our project 
This document describes the different parts of the program for
quadcopters developed on the Access Summer Project, 2015, and before,
in the Smart Mobility Lab, KTH.

% The hardware that we used 
The quadcopters used, and which this guide is aimed for, is the 3DR
IRIS$^+$. As a motion capture system Qualisys was used.

% About this guide
This guide provide information about how the program is structured,
Sec.~\ref{sec:architecture}, how ...


\section{Architecture}
\label{sec:architecture}

\begin{figure}[h!]                                                               
  \centering                                                                      
  \includegraphics[width=\textwidth]{figures/architecture}                             
  \caption{The architecture used in ROS.}
  \label{fig:architecture}                                                              
\end{figure}

The program is using ROS, \textit{The Robot
  Operating System}, to connect different parts of the program. A
basic knowledge of ROS is assumed. The architecture used in the program is shown in
Fig.~\ref{fig:architecture}. Circles does not correspond to ROS nodes
strictly, but describes the different parts of the
program. Rectangles, however, correspond to ROS topics. ROS packages
are also written in the figure. Arrows with black head corresponds to
publication to a topic, if the arrow points from a circle to a topic,
and subscription to a topic, if the arrow points from a topic to a
circle. Names in the following paragraphs are referring to
Fig.~\ref{fig:architecture}.

The goal of the program is to process commands from the user,
forwarded through the GUI, so that the quadcopter, either in the real
world or in the simulator, acts as the user
intends. The GUI can
start different ROS launch files, in turn staring scripts that publishes points
for the quadcopter to follow. These scripts are located in the
trajectory\textunderscore generator package, and for example publishes
points for a line or an arc. Also, the GUI can publish particular
points itself, not using a predefined script. For more information of
the GUI, see Sec.~\ref{sec:gui}. Apart from commands from the user, information about the quadcopter's
location, speed, acceleration, pitch, roll and yaw is also
needed. This is provided by Qualisys, taken into the ROS framework by
the mocap package.

A controller can be applied, since both the target point and the
current point is known. This is done in the controller package. The
current point is processed by the Security Guard. If not the
quadcopter is within certain safety limits, etc., the Lander node is
told to land the quadcopter by the Security Guard. If it is, the
Blender is told to further process the data. The blender got it's name
because it can ``blend'' outputs of different controllers and
collision avoidance. The outputs of the controllers are then
transformed to roll, pitch, throttle and yaw in the
Blender. This is then sent to Mavros on the topic
/irisX/mavros/rc/override/ (X $= 1, 2, 3, 4, \dots$) and finally to the
quadcopter.


\section{GUI}
\label{sec:gui}


\end{document}
